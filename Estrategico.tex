\documentclass{abnt}

%Arquivo com os principais pacotes usados e suas descrições.

%%%%%%%%%%%%%%%%%%%%%%%%%%%%%%%%%%%%%%%%%
% 			Idiomas e Acentos			%
%%%%%%%%%%%%%%%%%%%%%%%%%%%%%%%%%%%%%%%%%
\usepackage[brazilian]{babel} % Habilita o uso do idioma português do brasil (PT-BR).
\usepackage[T1]{fontenc} 
%\usepackage{fontspec} % Habilita maior variedade de acentos. Pode ser necessario adicionar outros pacotes.
\usepackage{lmodern} % Habilita o uso da font Latin Modern.


%%%%%%%%%%%%%%%%%%%%%%%%%%%%%%%%%%%%%%%%%
% 				TABELAS					%
%%%%%%%%%%%%%%%%%%%%%%%%%%%%%%%%%%%%%%%%%
\usepackage{tabulary} % Cria tabelas mais facilmente.
\usepackage{booktabs} % Melhora o visual das tabelas.
\usepackage[table]{xcolor} % Pacote de cor pra as tabelas.
\usepackage{caption} % Melhora as legendas de imagens, tabela etc.

%%%%%%%%%%%%%%%%%%%%%%%%%%%%%%%%%%%%%%%%%
% 				IMAGENS					%
%%%%%%%%%%%%%%%%%%%%%%%%%%%%%%%%%%%%%%%%%
\usepackage{graphicx} % Facilita a inserção de imagens.


%%%%%%%%%%%%%%%%%%%%%%%%%%%%%%%%%%%%%%%%%
% 			CÓDIGO FONTE				%
%%%%%%%%%%%%%%%%%%%%%%%%%%%%%%%%%%%%%%%%%

%Documentação de código fonte.
\usepackage{listings}


%%%%%%%%%%%%%%%%%%%%%%%%%%%%%%%%%%%%%%%%%
% 	Símbolos e Caracteres Matemáticos	%
%%%%%%%%%%%%%%%%%%%%%%%%%%%%%%%%%%%%%%%%%
\usepackage{amsmath}
\usepackage{amssymb}
\usepackage{amsfonts}
\usepackage{mathspec} %Habilita o uso das fontes e dos caracteres matematicos.


%%%%%%%%%%%%%%%%%%%%%%%%%%%%%%%%%%%%%%%%%
%				ABNT					%
%%%%%%%%%%%%%%%%%%%%%%%%%%%%%%%%%%%%%%%%%
\usepackage[alf, abnt-etal-cite=2]{abntcite} % Ordena as referencias em ordem alfabética.
\usepackage{url} %Facilita o uso de url. Pode-se usar o comando \url{...}.


%%%%%%%%%%%%%%%%%%%%%%%%%%%%%%%%%%%%%%%%%
% 			Configurações				%
%%%%%%%%%%%%%%%%%%%%%%%%%%%%%%%%%%%%%%%%%
\captionsetup{justification=centering,labelfont=bf} %Formata a legenda das figuras.
%\graphicspath{{../imgs/}} %Define o diretorio padrão para buscar as imagens da apresentação.  
%\setromanfont[Ligatures=TeX]{Crimson}
%\defaultfontfeatures{Scale=MatchLowercase, Mapping=tex-tex}

%%%%%%%%%%%%%%%%%%%%%%%%%%%%%%%%%%%%%%%%%
%				BEAMER					%
%%%%%%%%%%%%%%%%%%%%%%%%%%%%%%%%%%%%%%%%%
%Define algumas configurações que serão validas para todo o documento.  
%\setbeamertemplate{section in toc}[sections numbered]
%\setbeamertemplate{subsection in toc}[subsections numbered]
%\setbeamertemplate{background canvas}[vertical shading][bottom=blue!3,top=blue!7]
%\setbeamertemplate{caption}[numbered]


%%%%% Dados para criação da capa e folha de rosto %%%%
\autor{	%Denis F. de Carvalho,
	Guilherme A. de Macedo,
	Matheus L. Domingues da Silva e 
	Victor H. Carlquist da Silva
}
\titulo{Planejamento Estratégico}
\orientador{Avelino Natal Bazanella Junior}
\comentario{Trabalho apresentado ao Prof. Avelino Bazanela Junior, na disciplina de Administração 
			presente no $2^{o}$ modulo do curso de Tecnologia em Análise e Desenvolvimento de Sistemas no IFSP-CJO.}
\instituicao{Instituto Federal de Educação, Ciência e Tecnologia de São Paulo -- \textit{campus} Campos do Jordão}
\local{Campos do Jordão}
\data{\today}

\begin{document}

	% Para utilizar o formato padrão de capa da ABNT, substituí o comando \maketitle pelo comando \capa.
	\capa
	
	\folhaderosto

	\sumario
	
	\begin{resumo}
		Elaboração de um planejamento estratégico.
	\end{resumo}

	\chapter {Visão}
		Ser uma empresa de atuação e reconhecimento nacional, desenvolvedora
		de jogos inovadores, contribuindo para a diversão tanto de pessoas 
		adultas como no desenvolvimento intelectual das crianças de todo o Brasil.
	
	\chapter {Análise do Ambiente Externo}
		O cenário para jogos no Brasil é bastante promissor, vejamos a seguir
		qual a situação atual do mercado de jogos em 2011.

		\section {Situação atual do mercado de jogos no Brasil}
		O Brasil é composto por 200 milhões de pessoas, das quais 46 milhões 
		constituiem-se população ativa na internet, dessas 46 milhões de pessoas
		35 milhões são jogadores ativos.

		Pesquisas sobre o mercado de jogos indicam que existem três campos
		onde há muito mercado para ser explorado: em primeiro lugar estão os 
		jogadores de sites de jogos casuais; em segundo lugar os jogadores
		de redes sociais; e em terceiro lugar os jogadores de dispositivos
		móveis.

		\section {Situação do nicho específico em que atuaremos}
		Por ser um mercado emergente e os jogos serem de fácil disseminação, 
		nossa empresa se empenhará em desenvolver jogos para a plataforma de
		dispositivos móveis.

		Os jogos serão disponibilizados gratuitamente até uma determinada fase e
		com propagandas; o cliente, gostando do jogo, e se proponto a pagar, as 
		propagandas serão removidas e todas as fases serão desbloqueadas.

		\section {Tendências tecnológicas}
		O mercado de smartphones e tablets no brasil não para de crescer. 
		Levando em conta que os jogos são uma fonte de entretenimento inesgotável
		nesses dispositivos e imaginando que esse mercado não irá se saturar, 
		criaremos soluções de jogos inovadoras, simples e rentáveis.


	\chapter {Análise Interna}
		O ponto mais forte da nossa empresa é a mão de obra: temos bons designers,  
		bons programadores e boas ideias.

		A principal dificuldade será canalizar todas as energias dos integrantes 
		do nosso grupo para o caminho certo. Ou seja, integrar todas as qualidades
		dos integrantes da nossa equipe de modo que o trabalho final seja de
		qualidade e inovador.

	\chapter {Análise dos Concorrentes}
	
		Para a empresa ser competitiva é necessário oferecer produtos e/ou serviços do mesmo nível ou superior que se tem no mercado.
		Hoje há grandes empresas no seguimento de desenvolvimento de jogos \textit{mobile}. A Ubisoft\textsuperscript{\texttrademark} em parceria com a Gamesoft\textsuperscript{\texttrademark}, por exemplo, desenvolvem jogos com gráficos de alta qualidade, além de portar franquias de sucessos dos consoles para os \textit{mobiles}, como o Assassin's Creeds III\textsuperscript{\textregistered}.
	
		Os concorrentes geralmente usam \textit{engines} (motores gráficos) desenvolvidos pelas próprias empresas, dando caracteristicas únicas para os jogos.
		Mas há desenvolvedoras menores que fizeram muito sucesso, sem muito investimento, como é o caso da Rovio que desenvolveu Angry Birds\textsuperscript{\textregistered}, um jogo relativamente simples mas viciante. 
		
		Hoje, segundo a Business Insider, a Rovio\textsuperscript{\texttrademark} é a  23\textsuperscript{\underline{a}} empresa mais valiosa do mundo!
	Como podemos observar no mercado, o que mais conta é a criatividade.

	\chapter {Diferenciais Competitivos}
		Um diferencial competitivo seria ter bons roteiristas para criarem as histórias dos jogos, pois assim não seria necessário ter grande investimento em gráficos, já que os \textit{mobiles} não possuem grande poder de processamento gráfico.
		Um outro diferencial seria os usuários poderem votar em quais tipos de fases eles gostariam que o jogo possuísse.
	\chapter {Fatores Críticos de Sucesso}
		Um fator crítico para a empresa é manter o jogador sempre ativo, por meio de um jogo de qualidade e com boa história.
		É preciso de pessoas especializadas, motivadas e com talento dentro de suas áreas.
		Possuir um bom plano de marketing também é essencial, pois quem irá comprar o jogo sem ao menos conhecê-lo?. Um exemplo disso é a Ubisoft\textsuperscript{\texttrademark} que investiu US\$ 6,46 milhões para divulgar o Assassin's Creed III.
	
		Sempre é preciso ficar atento sobre as mudanças nas tecnologias, pois a plataforma \textit{mobile} está em constante mudança em poder de processamento e tamanho de tela, sendo possível usar o máximo dos recursos desses dispositivos. 

	\chapter {Definição da Missão}
	
		Nossa missão é proporcionar entretenimento as diversas classes sociais através 
		de jogos digitais de qualidade, com ótima performance e preços acessíveis.
	
	\chapter {Definição dos Valores}
	
		\begin{enumerate}
			\item Ética e respeito à dignidade, à liberdade intelectual e às diferenças;
			\item Busca da autonomia institucional com transparência e responsabilidade social;
			\item Compromisso social como:
				\subitem - facilitar acesso a meios de entretenimento;
				\subitem - exercício da gestão compartilhada com corresponsabilidade solidária;
				\subitem - desenvolvimento cultural, artístico, tecnológico e socioeconômico local, nacional e global.
		\end{enumerate}
			
		
	\chapter {Definição da Estratégia}
		\section{Objetivos Organizacionais}
		
		\begin{enumerate}
			\item Proporcionar acesso a culturas regionais, nacionais e internacionais através de jogos digitais;
			\item Proporcionar conforto e bem-estar dos funcionários em ambiente empresarial;
			\item Expansão da empresa em diversas outras unidades.
		\end{enumerate}

\end{document}

