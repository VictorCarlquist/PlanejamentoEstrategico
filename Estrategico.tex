\documentclass{abnt}

%Arquivo com os principais pacotes usados e suas descrições.

%%%%%%%%%%%%%%%%%%%%%%%%%%%%%%%%%%%%%%%%%
% 			Idiomas e Acentos			%
%%%%%%%%%%%%%%%%%%%%%%%%%%%%%%%%%%%%%%%%%
\usepackage[brazilian]{babel} % Habilita o uso do idioma português do brasil (PT-BR).
\usepackage[T1]{fontenc} 
%\usepackage{fontspec} % Habilita maior variedade de acentos. Pode ser necessario adicionar outros pacotes.
\usepackage{lmodern} % Habilita o uso da font Latin Modern.


%%%%%%%%%%%%%%%%%%%%%%%%%%%%%%%%%%%%%%%%%
% 				TABELAS					%
%%%%%%%%%%%%%%%%%%%%%%%%%%%%%%%%%%%%%%%%%
\usepackage{tabulary} % Cria tabelas mais facilmente.
\usepackage{booktabs} % Melhora o visual das tabelas.
\usepackage[table]{xcolor} % Pacote de cor pra as tabelas.
\usepackage{caption} % Melhora as legendas de imagens, tabela etc.

%%%%%%%%%%%%%%%%%%%%%%%%%%%%%%%%%%%%%%%%%
% 				IMAGENS					%
%%%%%%%%%%%%%%%%%%%%%%%%%%%%%%%%%%%%%%%%%
\usepackage{graphicx} % Facilita a inserção de imagens.


%%%%%%%%%%%%%%%%%%%%%%%%%%%%%%%%%%%%%%%%%
% 			CÓDIGO FONTE				%
%%%%%%%%%%%%%%%%%%%%%%%%%%%%%%%%%%%%%%%%%

%Documentação de código fonte.
\usepackage{listings}


%%%%%%%%%%%%%%%%%%%%%%%%%%%%%%%%%%%%%%%%%
% 	Símbolos e Caracteres Matemáticos	%
%%%%%%%%%%%%%%%%%%%%%%%%%%%%%%%%%%%%%%%%%
\usepackage{amsmath}
\usepackage{amssymb}
\usepackage{amsfonts}
\usepackage{mathspec} %Habilita o uso das fontes e dos caracteres matematicos.


%%%%%%%%%%%%%%%%%%%%%%%%%%%%%%%%%%%%%%%%%
%				ABNT					%
%%%%%%%%%%%%%%%%%%%%%%%%%%%%%%%%%%%%%%%%%
\usepackage[alf, abnt-etal-cite=2]{abntcite} % Ordena as referencias em ordem alfabética.
\usepackage{url} %Facilita o uso de url. Pode-se usar o comando \url{...}.


%%%%%%%%%%%%%%%%%%%%%%%%%%%%%%%%%%%%%%%%%
% 			Configurações				%
%%%%%%%%%%%%%%%%%%%%%%%%%%%%%%%%%%%%%%%%%
\captionsetup{justification=centering,labelfont=bf} %Formata a legenda das figuras.
%\graphicspath{{../imgs/}} %Define o diretorio padrão para buscar as imagens da apresentação.  
%\setromanfont[Ligatures=TeX]{Crimson}
%\defaultfontfeatures{Scale=MatchLowercase, Mapping=tex-tex}

%%%%%%%%%%%%%%%%%%%%%%%%%%%%%%%%%%%%%%%%%
%				BEAMER					%
%%%%%%%%%%%%%%%%%%%%%%%%%%%%%%%%%%%%%%%%%
%Define algumas configurações que serão validas para todo o documento.  
%\setbeamertemplate{section in toc}[sections numbered]
%\setbeamertemplate{subsection in toc}[subsections numbered]
%\setbeamertemplate{background canvas}[vertical shading][bottom=blue!3,top=blue!7]
%\setbeamertemplate{caption}[numbered]


%%%%% Dados para criação da capa e folha de rosto %%%%
\autor{	Denis F. de Carvalho,
	Guilherme A. de Macedo,
	Matheus L. Domingues da Silva e 
	Victor H. Carlquist da Silva
}
\titulo{Planejamento Estratégico}
\orientador{Avelino Natal Bazanella Junior}
\comentario{Trabalho apresentado ao Prof. Avelino Bazanela Junior, na disciplina de Administração 
			presente no $2^{a}$ modulo do curso de Tecnologia em Análise e Desenvolvimento de Sistemas no IFSP-CJO.}
\instituicao{Instituto Federal de Educação, Ciência e Tecnologia de São Paulo -- \textit{campus} Campos do Jordão}
\local{Campos do Jordão}
\data{\today}

\begin{document}

	% Para utilizar o formato padrão de capa da ABNT, substituí o comando \maketitle pelo comando \capa.
	\capa
	
	\folhaderosto

	\sumario
	
	\begin{resumo}
		Elaboração de um planejamento estratégico.
	\end{resumo}

	\chapter {Visão}
	
	\chapter {Análise do Ambiente Externo}

	\chapter {Análise Interna}

	\chapter {Análise dos Concorrentes}
	
	Para a empresa ser competitiva é necessário oferecer produtos e/ou serviços do mesmo nível ou superior que se tem no mercado.
	Hoje há grandes empresas no seguimento de desenvolvimento de jogos \textit{mobile}. A Ubisoft\texttrademark em parceria com a Gamesoft\texttrademark, por exemplo, desenvolvem jogos com gráficos de alta qualidade, além de portar franquias de sucessos dos consoles para os \textit{mobile}, como o Assassin's Creeds III\textregistered.
	
	Os concorrentes geralmente usam \textit(engines) (motores gráficos) desenvolvidas pelas próprias empresas, dando caracteristicas únicas para os jogos.
	Mas há desenvolvedoras menores que fizeram muito sucesso, sem muito investimento, como é o caso da Rovio que desenvolveu Angry Birds\textregistered, um jogo relativamente simples mas viciante. Hoje a Rovio é a  23\textsuperscript{\textun{a}} empresa mais valiosa do mundo!
	Como podemos observar o mercado, o que mais conta é a criatividade.

	\chapter {Diferenciais Competitivos}
	Um diferencial competitivo seria ter bons roteiristas para criarem as histórias dos jogos, pois assim não seria necessário ter grande investimento em gráficos, já que os \textit{mobiles} não possuem grande poder de processamento gráfico.
	
	Um outro diferencial seria os usuários poderem votar em quais tipos de fases eles gostariam que o jogo possuísse.
	\chapter {Fatores Críticos de Sucesso}
	Um fator crítico para a empresa é manter o jogador sempre ativo, por meio de um jogo de qualidade e com boa história.
	É preciso de pessoas especializadas, motivadas e com talento dentro de suas áreas.
	Possuir um bom plano de marketing também é essencial, pois quem irá comprar o jogo sem ao menos conhecê-lo?. Um exemplo disso é a Ubisoft\texttrademark que investiu US\$ 6,46 milhões para divulgar o Assassin's Creed III.
	
	Sempre é preciso ficar atento sobre as mudanças nas tecnologias, pois a plataforma \textit{mobile} está em constante mudança em poder de processamento e tamanho de tela, sendo possível usar o máximo dos recursos desses dispositivos. 
	
	\chapter {Definição da Missão}
	
	\chapter {Definição dos Valores}

	\chapter {Definição da Estratégia}

\end{document}
